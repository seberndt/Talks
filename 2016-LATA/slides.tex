\documentclass[german,table]{beamer}
\usepackage[utf8]{luainputenc}

\usepackage[english,german]{babel}

\usepackage{beamerthemeuzl-conference}
\usepackage{amsmath}
\usepackage{amssymb}
\usepackage{amsthm}
\usepackage{amsfonts}
\usepackage{marvosym}
\usepackage{subfigure}
\usepackage{mdframed}
\usepackage{bm}
\usepackage{ifthen}
\usepackage{relsize}
\usepackage{pifont}% http://ctan.org/pkg/pifont
\newcommand{\cmark}{\ding{51}}%
\newcommand{\xmark}{\ding{55}}%


\usetikzlibrary{decorations.pathmorphing, calc, positioning, backgrounds,
  fit, shapes, arrows.meta, decorations.pathreplacing, arrows,
  decorations.markings, patterns}
\DeclareMathAlphabet{\mathpzc}{OT1}{pzc}{m}{it}
\DeclareSymbolFont{missing}{OML}{cmr}{m}{n}
\DeclareMathSymbol{\ell}{\mathord}{missing}{'140}

\DeclareMathOperator{\SIZE}{SIZE}
\DeclareMathOperator{\poly}{poly}
\DeclareMathOperator{\Lang}{Lang}
\newcommand{\dea}{\ := \ }
\newcommand{\Expect}{{\rm I\kern-.3em E}}
\newcommand{\kla}{\mathrel{\ \le \ }}

\tikzset{onslide/.code args={<#1>#2}{%
    \only<#1>{\pgfkeysalso{#2}} % \pgfkeysalso doesn't change the path
  }}

\newenvironment{colortabular}[1]{\medskip\rowcolors[]{1}{black!10}{structure.fg!3}\tabular{#1}\rowcolor{black!20}}{\endtabular\medskip}

\newcommand{\tikzmark}[1]{\tikz[overlay,remember picture] \node (#1) {};}

% \tikzset{onslide/.code args={<#1>#2}{%
% \only<#1>{\pgfkeysalso{#2}} 
% }}

\definecolor{myred}{RGB}{228,32,50}
\definecolor{myblue}{RGB}{60,169,213}
\definecolor{myyellow}{RGB}{250,187,0}
\definecolor{mybrown}{RGB}{219,215,187}
\newcommand{\red}[1]{\textcolor{myred}{#1}}
\newcommand{\blue}[1]{\textcolor{myblue}{#1}}
\newcommand{\yellow}[1]{\textcolor{myyellow}{#1}}
\newcommand{\brown}[1]{\textcolor{mybrown}{#1}}

\usepackage{pdfcomment}
\newcommand{\pdfnote}[1]{\marginnote{\pdfcomment[icon=note]{#1}}}

\title{Steganography Based on Pattern Languages}
\author{\emph{Sebastian Berndt} \and Rüdiger Reischuk}
\institute{Institut für Theoretische Informatik,
  Universität zu Lübeck}
\date{16.03.2016}

% \setbeamerfont{alerted text}{shape=\normalfont}
% \setbeamercolor{alerted text}{fg=black}
% \setbeamercolor{itemize/enumerate body}{fg=black!60!white}
% \setbeamercolor{block body}{bg=black!5!white}



\pgfdeclarelayer{l1}
\pgfdeclarelayer{background}
\pgfsetlayers{l1,background,main}

\setbeamertemplate{blocks}[rounded][shadow=false]
% \includeonlyframes{current}
\date{}
\begin{document}
\maketitle

\begin{frame}
  \frametitle{And now for something completely different!}

  \begin{itemize}[<+->]
  \item Cryptography: Hide the content of a message\medskip
  \item Steganography: Hide that a message is transferred
  \end{itemize}
\end{frame}

\begin{frame}

  \frametitle{Sometimes, cryptography is not enough}
  \begin{center}
    \includegraphics[height=.8\textheight]{wikileaks}  
  \end{center}
\end{frame}



\begin{frame}
  \frametitle{Overview}
  \tableofcontents
  
\end{frame}

\section{Steganography}

\begin{frame}
  \frametitle{The Prisoners' Problem (not the Dilemma)}
  
  \begin{center}
    \begin{tikzpicture}
      \node[label=below:{Alice}] (alice) {\reflectbox{\includegraphics[scale=0.05]{alice2}}};

      \node[label=below:{Bob},right of = alice, xshift=5cm] (bob) {\includegraphics[scale=0.05]{bob}};
      

      \onslide<2->{    \draw[-,decorate,decoration=snake,very thick,color=blue] ($(alice)+(0.7cm,0)$) to ($(bob)-(0.7cm,0)$);
        \draw[-,decorate,decoration=snake,transform
        canvas={yshift=0.1cm},very thick,color=blue] ($(alice)+(0.7cm,0)$) to
        ($(bob)-(0.7cm,0)$);
        \node[color=blue] (d1) at ($(alice)+(1.7cm,-0.4cm)$)  {$d_{1}$};
        \node[color=blue] (d2) at ($(alice)+(2.2cm,-0.4cm)$)  {$d_{2}$};
        \node[color=blue] (ddots1) at ($(alice)+(2.7cm,-0.4cm)$)  {$\cdots$};
        \node[color=blue] (di) at ($(alice)+(3.2cm,-0.4cm)$)  {\only<-5>{$d_{i}^{\phantom{*}}$}\only<6->{$d_{i}^{*}$}};
        \node[color=blue] (ddots2) at ($(alice)+(3.7cm,-0.4cm)$)  {$\cdots$};
        \node[color=blue] (dl) at ($(alice)+(4.2cm,-0.4cm)$)  {$d_{\ell}$};

      }
      

      
      \onslide<3->{    \node[above left of = bob,yshift=1cm,xshift=-1.9cm, text=blue]  (channel)
        {Channel $\mathcal{C} =(D(\mathcal{C})\subseteq \Sigma^{*},P(\mathcal{C}))$};
        
        \draw[->,color=blue, very thick](channel) to ($(bob)-(2cm,-0.3cm)$);
      }

      \onslide<4->{
        \node[below right of = alice, yshift=-0.5cm,xshift=0.4cm] (message)
        {$m$};
        \draw[->,very thick, dashed] ($(alice)+(0.7cm,-0.5cm)$) to (message);
        \onslide<4>{\draw[->,very thick,dashed] (message) to ($(bob)+(-0.7,-0.5cm)$);}

      }



      \onslide<5->{
        \draw[->,very thick,dashed] (message) to (di);    
      }


      \onslide<7->{    \node[label=below:{\textcolor{red}{Warden}},right of = alice, xshift=2cm,yshift=-3cm]
        (warden) {\includegraphics[scale=0.03]{officer}};
      }
      \onslide<8->{
        \draw[->,very thick,color=red] (warden) to (d1);
        \draw[->,very thick,color=red] (warden) to (d2);
        \draw[->,very thick,color=red] (warden) to (di);
        \draw[->,very thick,color=red] (warden) to (dl);
      }
      \onslide<9->{
        \node (k) at ($(bob)!0.5!(alice)+(0,3cm)$) {key $K$};
        \draw[thick] (alice.north) |- (k.west);
        \draw[thick] (bob.north) |- (k.east);
      }
    \end{tikzpicture}
  \end{center}


\end{frame}


\begin{frame}
  \frametitle{Requirements of a stegosystem}
  \begin{itemize}
    \onslide<1->{\item Warden should not be able to distinguish $d_{i}$ from
      $d_{i}^{*}$ (\alert{Security})\medskip}
    \onslide<2->{\item Bob should be able to reconstruct the
      message (\alert{Reliability})\medskip}
    \onslide<3->{\item Alice and Bob should be computational
      feasible (\alert{Efficiency})\medskip}
    \onslide<4->{ \item Alice should get high transmission rate
      (\alert{Rate-efficiency})\\\onslide<5->{(Bounded by the
        \emph{channel-entropy $n$})}
      \medskip}
    \onslide<6->{ \item Should work for as much channels as possible (\alert{Applicability})}
  \end{itemize}

  % \onslide<5->{
  % \begin{block}{Examples}
  %   \begin{itemize}
  %   \item \onslide<5->{Alice returns a random sample
  %     document\onslide<6->{ (secure, efficient)}}
  %   \item \onslide<7->{Alice sends the message as it is
  %     \onslide<8->{(reliable, efficient, rate-efficient)}}
  %   \item \onslide<9->{Alice takes samples until one embeds
  %     the message\onslide<10->{(secure, reliable, rate-efficient)}}
  %   \end{itemize}
  % \end{block}
  % }  
\end{frame}

% \begin{frame}

%   \frametitle{Steganography}

% \end{frame}

\begin{frame}
  \frametitle{Universality}
  \begin{block}{Universal}
    \begin{itemize}
    \item  Stegosystem is \alert{universal} if it works for \emph{every} channel.
    \item<2->
      Such secure systems can embed only $\log n$ bits. \onslide<3->{Practical
        systems embed $\sqrt{n}$.}
    \item<4-> To embed \texttt{HELLO WORLD}, we need length $\geq
      2^{88}\approx 3\cdot 10^{12}\text{PB}\approx 10^{10}\cdot
      \mathop{Space}(\text{Facebook})$.
      % \item<4-> Longstanding Conjecture: Rate $\sqrt{n}$ is always achievable!
    \end{itemize}
  \end{block}

  \onslide<5->{
    \begin{block}{Task}
      Be more specific: Develop stegosystem for large channel-family $\mathcal{F}\,$!
    \end{block}
  }



\end{frame}

\begin{frame}
  \frametitle{Previous Work}
  \begin{colortabular}{cccp{31mm}}
    LB on Rate & UB on Rate & Channels &  Authors \\
    $\log(n)$ & \xmark & universal & Hopper et al. (2002)\pause
    \\
    $\log(n)$ & $\log(n)$ & universal & Dedi{\'c} et
    al. (2005)\pause \\
    $\sqrt{n}$ & \xmark & Monomials & Li\'{s}kiewicz et al. (2011)
    \pause \\
    $\sqrt{n}$ & \xmark & Pattern Languages & \textcolor{structure.fg}{this work}
  \end{colortabular}
  \onslide<5>{Much more systems exist, but none are provable secure!}

\end{frame}

\section{Pattern Languages}
\begin{frame}
  \frametitle{Monomials}
  \begin{columns}
    \begin{column}{.28\textwidth}
      \begin{tikzpicture}
        

        \node (t) {\begin{colortabular}{r}
            $X=\red{?}01\red{?}1\red{?}$\\
            \hline
            $\red{0}01\red{0}1\red{0}$\\
            $\red{0}01\red{0}1\red{1}$\\
            $\red{0}01\red{1}1\red{0}$\\
            $\red{0}01\red{1}1\red{1}$\\
            $\red{1}01\red{0}1\red{0}$\\
            $\red{1}01\red{0}1\red{1}$\\
            $\red{1}01\red{1}1\red{0}$\\
            $\red{1}01\red{1}1\red{1}$
          \end{colortabular}};
        \onslide<2->{
          \draw[decorate, decoration={brace},thick] ($(t.north east) + (0,-.7cm)$) --
          ($ (t.south east) + (0,.2cm) $)
          node[midway,xshift=.5cm] {$D(\mathcal{C})$};
        }

      \end{tikzpicture}
    \end{column}
    \begin{column}{.68\textwidth}
      \onslide<3->{
        \begin{block}{The system}
          \begin{enumerate}
          \item Partition the positions of $X$ into blocks $B_{1},\ldots,B_{b}$ via PRP
          \item Replace the different $\red{?}$'s such that $\sum_{x\in B_{i}} x=m_{i}$
          \end{enumerate}
        \end{block}}
      
      \onslide<4->{ \begin{block}{Restrictions}
          \begin{itemize}
          \item<4-> Simplest non-trivial language
          \item<5-> Simple cross product
          \end{itemize}
        \end{block}}

    \end{column}
  \end{columns}
\end{frame}

\begin{frame}
  \frametitle{Patterns}
  \begin{columns}
    \begin{column}{.28\textwidth}

      \begin{colortabular}{c}
        $\pi=\red{x_{1}}01\blue{x_{2}}1\red{x_{1}}$\\
        \hline
        $011$\\
        $\red{0}011\red{0}$\\
        $01\blue{0}1$\\
        $\red{0}01\blue{0}1\red{0}$\\
        \ldots\\
        $\red{1111}01\blue{00}1\red{1111}$\\
        \ldots
      \end{colortabular}
    \end{column}
    \begin{column}{.68\textwidth}
      \onslide<2->{
        \begin{block}{Advantages}
          \begin{itemize}
          \item<2-> More realistic (forms, websites etc.)
          \item<3-> Much larger class of languages
          \end{itemize}
        \end{block}
      }
    \end{column}
  \end{columns}
\end{frame}


\section{The stegosystem}

\begin{frame}
  \frametitle{Pattern Channels}
  \begin{itemize}[<+->]
  \item $D(\mathcal{C})\subseteq \Lang(\pi)$\medskip
  \item The pattern $\pi$ is known (or can be learned)\medskip
  \item The documents are of approximately same size\medskip
  \item Substitutions of variables are independent\medskip
  \end{itemize}

  \begin{center}
  \begin{tikzpicture}
    \node (a) {};
    \node[right = of a, xshift=8cm] (b) {};
    \draw[-,decorate,decoration=snake,transform
    canvas={yshift=0.1cm},very thick,color=blue] (a) to
    (b);
    \draw[-,decorate,decoration=snake,very thick,color=blue] (a) to (b);
        \node[color=blue] (d1) at ($(a)+(1.7cm,-0.4cm)$)  {$d_{1}$};
        \node[color=blue] (d2) at ($(a)+(2.2cm,-0.4cm)$)  {$d_{2}$};
        \node[color=blue] (ddots1) at ($(a)+(2.7cm,-0.4cm)$)  {$\cdots$};
        \node[color=blue] (di) at ($(a)+(3.2cm,-0.4cm)$)  {$d_{i}$};
        \node[color=blue] (ddots2) at ($(a)+(3.7cm,-0.4cm)$)  {$\cdots$};
        \node[color=blue] (dl) at ($(a)+(4.2cm,-0.4cm)$)  {$d_{\ell}$};
        \node[color=blue] (symb) at ($(a)+(5.4cm,-0.4cm)$)  {$\sim \Lang(\pi)$};

  \end{tikzpicture}
  \end{center}

\end{frame}

\begin{frame}
  \frametitle{The stegosystem}
  \begin{itemize}[<+->]
  \item Expand $\pi = x_{1}01x_{2}1x_{1}$ into
    $[\pi]=y_{1}y_{2}y_{3}01y_{4}1y_{1}y_{2}y_{3}$.
  \item Use pseudo-random function $F_{K}$ to partition $[\pi]$ in $b$ Blocks

    \begin{center}
      

      \scalebox{.7}{
        \begin{tikzpicture}[node distance=0.03cm]
          \node[circle,onslide=<6->{draw,fill=myred}] (y1)
          {\only<-6>{$y_{1}$}\only<7->{\phantom{$y_{1}$}}};
          \node[below= of y1,circle,onslide=<6->{draw,fill=myblue}] (y2)
          {\only<-6>{$y_{2}$}\only<7->{\phantom{$y_{2}$}}};
          \node[below= of y2,circle,onslide=<6->{draw,fill=myyellow}] (y3) {\only<-6>{$y_{3}$}\only<7->{\phantom{$y_{3}$}}};
          \node[below= of y3] (t0) {\only<-6>{$0$}\only<7->{\phantom{0}}};
          \node[below= of t0] (t1) {\only<-6>{$1$}\only<7->{\phantom{1}}};
          \node[below= of t1,circle,onslide=<6->{draw,fill=mybrown}] (y4) {\only<-6>{$y_{4}$}\only<7->{\phantom{$y_{4}$}}};
          \node[below= of y4] (s1) {\only<-6>{$1$}\only<7->{\phantom{1}}};
          \node[below= of s1,circle,onslide=<6->{draw,fill=myred}] (y1p) {\only<-6>{$y_{1}$}\only<7->{\phantom{$y_{1}$}}};
          \node[below= of y1p,circle,onslide=<6->{draw,fill=myblue}] (y2p) {\only<-6>{$y_{2}$}\only<7->{\phantom{$y_{2}$}}};
          \node[below= of y2p,circle,onslide=<6->{draw,fill=myyellow}] (y3p) {\only<-6>{$y_{3}$}\only<7->{\phantom{$y_{3}$}}};

          \node[right = of y1, rectangle, draw, minimum
          width=4cm,xshift=3cm, minimum height=1cm] (b1) {
            \only<4-6>{$0$, $y_{4}$, $1$, $y_{1}$, $y_{3}$}
            \only<7->{ \tikz{\node[circle,draw,fill=mybrown,scale=.1]{}}
              \tikz{\node[circle,draw,fill=myred,scale=.1]{}}
              \tikz{\node[circle,draw,fill=myyellow,scale=.1]{}}
            }
          };
          \node[below = of b1, rectangle, draw, minimum width=4cm,minimum
          height=1cm,yshift=-1.6cm] (b2) {
            \only<4-6>{$y_{1}$, $y_{2}$}
            \only<7->{ 
              \tikz{\node[circle,draw,fill=myred,scale=.1]{}}
              \tikz{\node[circle,draw,fill=myblue,scale=.1]{}}
            }
          };
          \node[below = of b2, rectangle, draw, minimum width=4cm,minimum
          height=1cm,yshift=-1.6cm] (b3) {
            \only<4-6>{$y_{3}$, $1$, $y_{2}$}
            \only<7->{ 
              \tikz{\node[circle,draw,fill=myyellow,scale=.1]{}}
              \tikz{\node[circle,draw,fill=myblue,scale=.1]{}}
            }
          };

          \onslide<3->{
            \node[right = of y1, xshift=1.5cm,yshift=.2cm] {$F_{K}$};
            
            \draw[->,thick] (y1) to (b2.west);
            \draw[->,thick] (y2) to (b2.west);
            \draw[->,thick] (y3) to (b3.west);
            \draw[->,thick,onslide=<5->{opacity=0}] (t0) to (b1.west);
            \draw[->,thick,onslide=<5->{opacity=0}] (t1) to (b3.west);
            \draw[->,thick] (y4) to (b1.west);
            \draw[->,thick,onslide=<5->{opacity=0}] (s1) to (b1.west);
            \draw[->,thick] (y1p) to (b1.west);
            \draw[->,thick] (y2p) to (b3.west);
            \draw[->,thick] (y3p) to (b1.west);}

          \onslide<5>{
            \node[right = of b1] {$y_{1}+y_{3}+y_{4}+1=m_{1}$};
            \node[right = of b2] {$y_{1}+y_{2}=m_{2}$};
            \node[right = of b3] {$y_{2}+y_{3}+1=m_{3}$};

          }
        \end{tikzpicture}}
    \end{center}
  \end{itemize}


\end{frame}

\begin{frame}
  \frametitle{Coloured Balls into Bins}
  \begin{center}
    
    \begin{tikzpicture}
      
      \node[rectangle, draw, minimum
      width=4cm,minimum height=1cm] (b1) {
        \tikz{\node[circle,draw,fill=mybrown,scale=.1]{}}
        \tikz{\node[circle,draw,fill=myred,scale=.1]{}}
        \tikz{\node[circle,draw,fill=myyellow,scale=.1]{}}
      };
      
      \node[below = of b1, rectangle, draw, minimum width=4cm,minimum
      height=1cm,yshift=.8cm] (b2) {
        \tikz{\node[circle,draw,fill=myred,scale=.1]{}}
        \tikz{\node[circle,draw,fill=myblue,scale=.1]{}}
      };
      \node[below = of b2, rectangle, draw, minimum width=4cm,minimum
      height=1cm,yshift=.8cm] (b3) {
        \tikz{\node[circle,draw,fill=myyellow,scale=.1]{}}
        \tikz{\node[circle,draw,fill=myblue,scale=.1]{}}
      };

      \onslide<2->{
        
        \node[right = of b2] (tab){
          \begin{tabular}{c|c|c|c}
            \tikz{\node[circle,draw,fill=mybrown,]{}}&
                                                       \tikz{\node[circle,draw,fill=myred,]{}}&
                                                                                                \tikz{\node[circle,draw,fill=myyellow,]{}}&
                                                                                                                                            \tikz{\node[circle,draw,fill=myblue,]{}}\\
            \hline
            1 & 1 & 1 & 0\\
            0 & 1 & 0 & 1\\
            0 & 0 & 1 & 1
          \end{tabular}
        };
        
        \draw[thick,-{>[]}] (b1) to (tab);
        \draw[thick,-{>[]}] (b2) to (tab);
        \draw[thick,-{>[]}] (b3) to (tab);
      }
    \end{tikzpicture}
  \end{center}
  \begin{itemize}
  \item<3-> Columns are independent
  \item<4-> Rows are \emph{not} independent
  \end{itemize}

  \onslide<5->{
    \begin{block}{Theorem (Generalization of Poisson Approximation)}
      W.h.p. there are many colours such that the corresponding rows behave independently!
    \end{block}}


\end{frame}

\begin{frame}
  \frametitle{Some Useful Lemmata}

  \begin{lemma}
    Let $C$ be the coloured-balls-into-bin matrix with $\mu$ bins, where
    each colour appears at most $\xi$ times and let $X$ be a matrix of
    pairwise independent Poisson-distributed RVs.

    For every predicate $Q$ of arity $l$, let $\mathcal{E}_{Q}(X)$ be the
    probability that $X$ has a subset of $l$ columns
    $X_{z_{1}}, \ldots,X_{z_{l}}$ such that
    $Q(X_{z_{1}}, \ldots,X_{z_{l}})$ holds. Then:
    \begin{align*}
      \Pr[\mathcal{E}_{Q}(C)] \kla
      \frac{\Pr[\mathcal{E}_{Q}(X)]}{1-2\exp\left(-2 \; \eta(l,n,\xi)\right)} \ . 
    \end{align*}

  \end{lemma}
  \onslide<2->{
    \begin{lemma}
      The matrix $M$ with $m_{i,j}=x_{i,j} \bmod 2$  has full rank 
      over $\mathbb{F}_{2}$ with  probability at least $1-\mu\cdot (4/5)^{\mu}$.
    \end{lemma}}

  \onslide<3->{
    \begin{corollary}
      If \alert{$n \geq b^{2}$}, it holds that:
      $|\Pr[\mathcal{E}_{Q}(C)]-\Pr[\mathcal{E}_{Q}(X)]|\leq
      \exp(-n)$. 
    \end{corollary}}
  
\end{frame}




\section{Conclusion}
\begin{frame}
  \frametitle{Open Questions}
  \begin{itemize}[<+->]
  \item Public-Key Scenario?
  \item Steganography for other families of languages ($D(\mathcal{C})$
    described by automata, grammars, logics, \ldots)?
  \end{itemize}
\end{frame}

\begin{frame}
  \frametitle{Conclusion}

  \begin{center}
    {\Large Secure, rate-efficient steganography is possible on realistic channels!}  
  \end{center}

  \begin{center}
    
    \scalebox{.8}{
      \begin{tikzpicture}
        \node[label=below:{Alice}] (alice) {\reflectbox{\includegraphics[scale=0.05]{alice2}}};

        \node[label=below:{Bob},right of = alice, xshift=5cm] (bob) {\includegraphics[scale=0.05]{bob}};
        

        \draw[-,decorate,decoration=snake,very thick,color=blue] ($(alice)+(0.7cm,0)$) to ($(bob)-(0.7cm,0)$);
        \draw[-,decorate,decoration=snake,transform
        canvas={yshift=0.1cm},very thick,color=blue] ($(alice)+(0.7cm,0)$) to
        ($(bob)-(0.7cm,0)$);
        \node[color=blue] (d1) at ($(alice)+(1.7cm,-0.4cm)$)  {$d_{1}$};
        \node[color=blue] (d2) at ($(alice)+(2.2cm,-0.4cm)$)  {$d_{2}$};
        \node[color=blue] (ddots1) at ($(alice)+(2.7cm,-0.4cm)$)  {$\cdots$};
        \node[color=blue] (di) at ($(alice)+(3.2cm,-0.4cm)$)  {$d_{i}^{*}$};
        \node[color=blue] (ddots2) at ($(alice)+(3.7cm,-0.4cm)$)  {$\cdots$};
        \node[color=blue] (dl) at ($(alice)+(4.2cm,-0.4cm)$)  {$d_{\ell}$};

        

        
        \node[above left of = bob,yshift=1cm,xshift=-1.9cm, text=blue]  (channel)
        {Channel $\mathcal{C} =(D(\mathcal{C})\subseteq \Sigma^{*},P(\mathcal{C}))$};
        
        \draw[->,color=blue, very thick](channel) to ($(bob)-(2cm,-0.3cm)$);


        \node[below right of = alice, yshift=-0.5cm,xshift=0.4cm] (message)
        {$m$};
        \draw[->,very thick, dashed] ($(alice)+(0.7cm,-0.5cm)$) to (message);




        \draw[->,very thick,dashed] (message) to (di);    


        \node[label=below:{\textcolor{red}{Warden}},right of = alice, xshift=2cm,yshift=-3cm]
        (warden) {\includegraphics[scale=0.03]{officer}};


        \begin{scope}[on background layer]
          \draw[->,very thick,color=red] (warden) to (d1);
          \draw[->,very thick,color=red] (warden) to (d2);
          \draw[->,very thick,color=red] (warden) to (di);
          \draw[->,very thick,color=red] (warden) to (dl);
        \end{scope}


        \node (k) at ($(bob)!0.5!(alice)+(0,3cm)$) {key $K$};
        \draw[thick] (alice.north) |- (k.west);
        \draw[thick] (bob.north) |- (k.east);

      \end{tikzpicture}}
  \end{center}



\end{frame}
\end{document}



